%%%%%%%%%%%%%%%%%%%%%%%%%%%%%%%%%%%%%%%%%%%%%%%%%%%%%%%%%%%%%%%
%
% Welcome to writeLaTeX --- just edit your LaTeX on the left,
% and we'll compile it for you on the right. If you give
% someone the link to this page, they can edit at the same
% time. See the help menu above for more info. Enjoy!
%
%%%%%%%%%%%%%%%%%%%%%%%%%%%%%%%%%%%%%%%%%%%%%%%%%%%%%%%%%%%%%%%

% --------------------------------------------------------------
% This is all preamble stuff that you don't have to worry about.
% Head down to where it says "Start here"
% --------------------------------------------------------------
 
\documentclass[12pt]{article}
 
\usepackage[margin=1in]{geometry}
\usepackage{amsmath,amsthm,amssymb}

\usepackage{listings}
\usepackage{xcolor}
\usepackage{circuitikz}
\usetikzlibrary{bending}
\usetikzlibrary{patterns,decorations.pathmorphing,positioning}
\usepackage{enumitem}

\usepackage{tikz}
\usetikzlibrary{shapes,arrows,positioning,calc}
\tikzset{
block/.style = {draw, fill=white, rectangle, minimum height=3em, minimum width=3em},
tmp/.style  = {coordinate}, 
sum/.style= {draw, fill=white, circle},
input/.style = {coordinate},
output/.style= {coordinate},
pinstyle/.style = {pin edge={to-,thin,black}
}
}

\usepackage{float,graphicx}

%New colors defined below
\definecolor{codegreen}{rgb}{0,0.6,0}
\definecolor{codegray}{rgb}{0.5,0.5,0.5}
\definecolor{codepurple}{rgb}{0.58,0,0.82}
\definecolor{backcolour}{rgb}{0.95,0.95,0.92}

%Code listing style named "mystyle"
\lstdefinestyle{mystyle}{
  backgroundcolor=\color{backcolour}, commentstyle=\color{codegreen},
  keywordstyle=\color{magenta},
  numberstyle=\tiny\color{codegray},
  stringstyle=\color{codepurple},
  basicstyle=\ttfamily\footnotesize,
  breakatwhitespace=false,         
  breaklines=true,                 
  captionpos=b,                    
  keepspaces=true,                 
  numbers=left,                    
  numbersep=5pt,                  
  showspaces=false,                
  showstringspaces=false,
  showtabs=false,                  
  tabsize=2
}

%"mystyle" code listing set
\lstset{style=mystyle}

 
\newcommand{\N}{\mathbb{N}}
\newcommand{\Z}{\mathbb{Z}}
 
\newenvironment{theorem}[2][Theorem]{\begin{trivlist}
\item[\hskip \labelsep {\bfseries #1}\hskip \labelsep {\bfseries #2.}]}{\end{trivlist}}
\newenvironment{lemma}[2][Lemma]{\begin{trivlist}
\item[\hskip \labelsep {\bfseries #1}\hskip \labelsep {\bfseries #2.}]}{\end{trivlist}}
\newenvironment{exercise}[2][Exercise]{\begin{trivlist}
\item[\hskip \labelsep {\bfseries #1}\hskip \labelsep {\bfseries #2.}]}{\end{trivlist}}
\newenvironment{problem}[2][Problem]{\begin{trivlist}
\item[\hskip \labelsep {\bfseries #1}\hskip \labelsep {\bfseries #2.}]}{\end{trivlist}}
\newenvironment{question}[2][Question]{\begin{trivlist}
\item[\hskip \labelsep {\bfseries #1}\hskip \labelsep {\bfseries #2.}]}{\end{trivlist}}
\newenvironment{corollary}[2][Corollary]{\begin{trivlist}
\item[\hskip \labelsep {\bfseries #1}\hskip \labelsep {\bfseries #2.}]}{\end{trivlist}}

\newenvironment{solution}{\begin{proof}[Solution]}{\end{proof}}
 
\begin{document}
 
% --------------------------------------------------------------
%                         Start here
% --------------------------------------------------------------
 
\title{Homework 5}%replace X with the appropriate number
\author{Mengxiang Jiang\\ %replace with your name
EEEN 5338 Digital and DSP Based Control} %if necessary, replace with your course title
 
\maketitle
 
\begin{problem}{1} %You can use theorem, exercise, problem, or question here.  Modify x.yz to be whatever number you are proving
    Draw a flowchart from which the following compensator can be programmed if the sampling interval is 0.1 seconds.
    $$ G_c(z) = \frac{1899z^2-3761z+1861}{z^2-1.908z+0.9075}$$
    \begin{align*}
        G_c(z) = \frac{X(z)}{E(z)} = \frac{1899z^2-3761z+1861}{z^2-1.908z+0.9075}\\
        \text{Cross-multiply and obtain}\\
        (z^2-1.908z+0.9075)X(z) = (1899z^2-3761z+1861)E(z)\\
        \text{Solve for the highest power of $z$ operating on the output, $X(z)$,}\\
        z^2X(z) = (1899z^2-3761z+1861)E(z) - (-1.908z + 0.9075)X(z)\\
        \text{Solving for X(z) on the left-hand side,}\\
        X(z) = (1899-3761z^{-1}+1861z^{-2})E(z) - (-1.908z^{-1}+0.9075z^{-2})X(z)\\
        \text{Taking the inverse $z$-transform and substituting in $T=0.1$,}\\
        x^*(t) = 1899e^*(t) - 3761e^*(t-0.1) + 1861e^*(t-0.2) + 1.908x^*(t-0.1) - 0.9075x^*(t-0.2)
    \end{align*}

    \begin{tikzpicture}[auto, node distance=2.5cm]
        \node [input] (ein) {};
        \node [tmp, right of=ein] (tmp1) {};
        \node [block, right of=tmp1] (a2b2) {1899};
        \node [tmp, right of=a2b2] (tmp2) {};
        \node [block, below of=tmp1] (delay1) {Delay 0.1};
        \node [tmp, below of=delay1] (tmp3) {};
        \node [block, right of=tmp3] (a1b2) {3761};
        \node [sum, right of=a1b2] (sum1) {};
        \node [block, below of=tmp3] (delay2) {Delay 0.1};
        \node [tmp, below of=delay2] (tmp4) {};
        \node [block, right of=tmp4] (a0b2) {1861};
        \node [tmp, right of=a0b2] (tmp5) {};
        \node [sum, right of=sum1, node distance=1cm] (sum2) {}; 
        \node [block, right of=sum2] (b1b2) {1.908};
        \node [tmp, right of=b1b2] (tmp6) {};
        \node [block, above of=tmp6] (delay3) {Delay 0.1};
        \node [tmp, above of=delay3] (tmp7) {};
        \node [output, right of=tmp7] (xout) {};
        \node [block, below of=tmp6] (delay4) {Delay 0.1};
        \node [tmp, below of=delay4] (tmp8) {};
        \node [block, left of=tmp8] (b0b2) {0.9075};
        \node [tmp, left of=b0b2] (tmp9) {};

        \draw [-] (ein) -- node{$e^*(t)$} (tmp1);
        \draw [->] (tmp1) -- (a2b2);
        \draw [-] (a2b2) --  (tmp2);
        \draw [->] (tmp2) -- node{+} (sum1);
        \draw [->] (tmp1) -- (delay1);
        \draw [-] (delay1) -- (tmp3);
        \draw [->] (tmp3) -- node{$e^*(t-0.1)$} (a1b2);
        \draw [->] (a1b2) -- node{-} (sum1);
        \draw [->] (tmp3) -- (delay2);
        \draw [-] (delay2) -- (tmp4);
        \draw [->] (tmp4) -- node{$e^*(t-0.2)$} (a0b2);
        \draw [-] (a0b2) -- (tmp5);
        \draw [->] (tmp5) -- node{+} (sum1);
        \draw [->] (sum1) -- node{+} (sum2);
        \draw [->] (b1b2) -- node{+} (sum2);
        \draw [->] (tmp6) -- node{$x^*(t-0.1)$} (b1b2);
        \draw [-] (delay3) -- (tmp6);
        \draw [->] (tmp7) -- (delay3);
        \draw [->] (tmp7) -- node{$x^*(t)$} (xout);
        \draw [-] (tmp7) -| (sum2);
        \draw [->] (tmp6) -- (delay4);
        \draw [-] (delay4) -- (tmp8);
        \draw [->] (tmp8) -- node{$x^*(t-0.2)$} (b0b2);
        \draw [-] (b0b2) -- (tmp9);
        \draw [->] (tmp9) -- node{-} (sum2);

        % \node [output, right of=adc, node distance=2cm] (output) {};
        % \node [tmp, below of=sum1] (tmp1) {};
        % \draw [->] (ein) -- node{$e*(t)$} (sum1);
        % \draw [->] (sum1) --node{$E(z)$} (c_z);
        % \draw [->] (c_z) --node{$U(z)$} (dac);
        % \draw [->] (dac) -- (g_s);
        % \draw [->] (g_s) -- (adc);
        % \draw [->] (adc) -- node [name=y] {$Y(z)$}(output);
        % \draw [->] (y) |- (tmp1)-| (sum1);
        \end{tikzpicture}
\end{problem}

% --------------------------------------------------------------
%     You don't have to mess with anything below this line.
% --------------------------------------------------------------
 
\end{document}