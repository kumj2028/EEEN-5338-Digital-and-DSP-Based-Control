%%%%%%%%%%%%%%%%%%%%%%%%%%%%%%%%%%%%%%%%%%%%%%%%%%%%%%%%%%%%%%%
%
% Welcome to writeLaTeX --- just edit your LaTeX on the left,
% and we'll compile it for you on the right. If you give
% someone the link to this page, they can edit at the same
% time. See the help menu above for more info. Enjoy!
%
%%%%%%%%%%%%%%%%%%%%%%%%%%%%%%%%%%%%%%%%%%%%%%%%%%%%%%%%%%%%%%%

% --------------------------------------------------------------
% This is all preamble stuff that you don't have to worry about.
% Head down to where it says "Start here"
% --------------------------------------------------------------
 
\documentclass[12pt]{article}
 
\usepackage[margin=1in]{geometry}
\usepackage{amsmath,amsthm,amssymb}

\usepackage{listings}
\usepackage{xcolor}
\usepackage{circuitikz}
\usetikzlibrary{bending}
\usetikzlibrary{patterns,decorations.pathmorphing,positioning}
\usepackage{enumitem}

%New colors defined below
\definecolor{codegreen}{rgb}{0,0.6,0}
\definecolor{codegray}{rgb}{0.5,0.5,0.5}
\definecolor{codepurple}{rgb}{0.58,0,0.82}
\definecolor{backcolour}{rgb}{0.95,0.95,0.92}

%Code listing style named "mystyle"
\lstdefinestyle{mystyle}{
  backgroundcolor=\color{backcolour}, commentstyle=\color{codegreen},
  keywordstyle=\color{magenta},
  numberstyle=\tiny\color{codegray},
  stringstyle=\color{codepurple},
  basicstyle=\ttfamily\footnotesize,
  breakatwhitespace=false,         
  breaklines=true,                 
  captionpos=b,                    
  keepspaces=true,                 
  numbers=left,                    
  numbersep=5pt,                  
  showspaces=false,                
  showstringspaces=false,
  showtabs=false,                  
  tabsize=2
}

%"mystyle" code listing set
\lstset{style=mystyle}

 
\newcommand{\N}{\mathbb{N}}
\newcommand{\Z}{\mathbb{Z}}
 
\newenvironment{theorem}[2][Theorem]{\begin{trivlist}
\item[\hskip \labelsep {\bfseries #1}\hskip \labelsep {\bfseries #2.}]}{\end{trivlist}}
\newenvironment{lemma}[2][Lemma]{\begin{trivlist}
\item[\hskip \labelsep {\bfseries #1}\hskip \labelsep {\bfseries #2.}]}{\end{trivlist}}
\newenvironment{exercise}[2][Exercise]{\begin{trivlist}
\item[\hskip \labelsep {\bfseries #1}\hskip \labelsep {\bfseries #2.}]}{\end{trivlist}}
\newenvironment{problem}[2][Problem]{\begin{trivlist}
\item[\hskip \labelsep {\bfseries #1}\hskip \labelsep {\bfseries #2.}]}{\end{trivlist}}
\newenvironment{question}[2][Question]{\begin{trivlist}
\item[\hskip \labelsep {\bfseries #1}\hskip \labelsep {\bfseries #2.}]}{\end{trivlist}}
\newenvironment{corollary}[2][Corollary]{\begin{trivlist}
\item[\hskip \labelsep {\bfseries #1}\hskip \labelsep {\bfseries #2.}]}{\end{trivlist}}

\newenvironment{solution}{\begin{proof}[Solution]}{\end{proof}}
 
\begin{document}
 
% --------------------------------------------------------------
%                         Start here
% --------------------------------------------------------------
 
\title{Homework 3}%replace X with the appropriate number
\author{Mengxiang Jiang\\ %replace with your name
EEEN 5338 Digital and DSP Based Control} %if necessary, replace with your course title
 
\maketitle
 
\begin{problem}{1} %You can use theorem, exercise, problem, or question here.  Modify x.yz to be whatever number you are proving
    Many chemical processes can be modeled by the following transfer function:
    $$G(s) = \frac{K}{\tau s+1}e^{-T_{d}s}$$
    where $K$ is the gain, $\tau$ is the time constant, and $T_d$ is the time delay. Obtain the transfer function $G_{ZAS}(z)$ for the system in terms of the system parameters. Assume that the time delay $T_d$ is a multiple of the sampling period $T$.
    \begin{enumerate}[label=\alph*.]
      \item Partial fraction expansion:
      \begin{align*}
        \frac{G(s)}{s} = \frac{K}{s(\tau s+1)}e^{-T_ds} = \left(\frac{A}{s}+\frac{B}{\tau s+1}\right)e^{-T_ds}\\
        K = (\tau s+1)A + sB\\
        s=0\implies K = A\\
        s=-\frac{1}{\tau} \implies K= -\frac{B}{\tau} \implies B = -\tau K\\
        \implies \frac{G(s)}{s} = \left(\frac{K}{s} - \frac{\tau K}{\tau s + 1}\right)e^{-T_ds} = \left(\frac{K}{s} - \frac{K}{s + 1/\tau}\right)e^{-T_ds} 
      \end{align*} 
      \item $z$-transfer function:
      \begin{align*}
        G_{ZAS}(z) = (1-z^{-1})\mathcal{Z}\left\{\frac{G(s)}{s}\right\}=(1-z^{-1})\mathcal{Z}\left\{\left(\frac{K}{s}-\frac{K}{s+1/\tau}\right)e^{-T_ds}\right\}\\
        = (1-z^{-1})\left(\frac{Kz}{z-1}-\frac{Kz}{z-e^{-T/\tau}}\right)z^{-T_d/T}\\
        \text{using Wolfram Alpha to simplify the algebra:}\\
        G_{ZAS}(z)=\frac{(e^{-T/\tau}-1)K}{e^{-T/\tau}-z}z^{-T_d/T}
      \end{align*} 
  \end{enumerate}
\end{problem}
\pagebreak
\begin{problem}{2}
  For an internal combustion engine, the transfer function with injected fuel rate as input and fuel flow rate into the cylinder as output is given by:
  $$G(s) = \frac{\varepsilon\tau s + 1}{\tau s + 1}$$
  where $\tau$ is a time constant and $\varepsilon$ is known as the fuel split parameter. Obtain the transfer function $G_{ZAS}(z)$ for the system in terms of the system parameters.
  \begin{enumerate}[label=\alph*.]
    \item Partial fraction expansion:
    \begin{align*}
      \frac{G(s)}{s} =\frac{\varepsilon\tau s + 1}{s(\tau s + 1)} = \frac{A}{s}+\frac{B}{\tau s+1}\\
      \varepsilon\tau s + 1 = (\tau s+1)A + sB\\
      s=0\implies 1 = A\\
      s=-\frac{1}{\tau} \implies -\varepsilon + 1= -\frac{B}{\tau} \implies B = \varepsilon\tau - \tau\\
      \implies \frac{G(s)}{s} = \frac{1}{s} + \frac{\varepsilon\tau-\tau}{\tau s + 1} = \frac{1}{s} + \frac{\varepsilon - 1}{s + 1/\tau}
    \end{align*} 
    \item $z$-transfer function:
    \begin{align*}
      G_{ZAS}(z) = (1-z^{-1})\mathcal{Z}\left\{\frac{G(s)}{s}\right\}=(1-z^{-1})\mathcal{Z}\left\{\frac{1}{s} + \frac{\varepsilon - 1}{s + 1/\tau}\right\}\\
      = (1-z^{-1})\left(\frac{z}{z-1}+\frac{z(\varepsilon-1)}{z-e^{-T/\tau}}\right)\\
      \text{using Wolfram Alpha to simplify the algebra:}\\
      G_{ZAS}(z)=1+\frac{(\varepsilon-1)(z-1)}{z-e^{-T/\tau}}
    \end{align*} 
\end{enumerate}
\end{problem}
\pagebreak
\begin{problem}{3}
Repeat \textbf{Problem 2} including a delay of 25 ms in the transfer function with a sampling period of 10 ms.
\begin{enumerate}[label=\alph*.]
  \item Transport delay:
  \begin{align*}
    T_d = lT - mT, \text{ where $l$ is a positive integer and $0\leq m < 1$}\\
    25 = l(10) - m(10) = 3(10) - 0.5(10)\\
    \implies l=3,\; m=0.5
  \end{align*} 
  \item $z$-transfer function:
  \begin{align*}
    G_{ZAS}(z) = (1-z^{-1})\left(\frac{z}{z-1} + \frac{z(\varepsilon-1)e^{-5/\tau}}{z-e^{-T/\tau}}\right)z^{-3}\\
    \text{using Wolfram Alpha to simplify the algebra:}\\
    G_{ZAS}(z)=\left(1+\frac{(\varepsilon-1)(z-1)e^{-5/\tau}}{z-e^{-T/\tau}}\right)z^{-3}
  \end{align*} 
\end{enumerate}
\end{problem}
\pagebreak
\begin{problem}{4}
  Find the equivalent sampled impulse response sequence and the equivalent $z$-transfer function for the cascade of the two analog systems with sampled input
  $$H_1(s) = \frac{1}{s+6}$$
  $$H_2(s) = \frac{10}{s+1}$$
  \begin{enumerate}[label=\alph*.]
    \item If the systems are directly connected
    \begin{align*}
      H(s) = H_1(s)H_2(s) = \frac{10}{(s+6)(s+1)} = \frac{A}{s+6} + \frac{B}{s+1}\\
      10 = (s+1)A + (s+6)B\\
      s= -6 \implies 10 = -5A \implies A = -2\\
      s=-1 \implies 10 = 5B \implies B = 2\\
      H(s) = \frac{-2}{s+6} + \frac{2}{s+1}\\
      h(t) = -2e^{-6t} + 2e^{-t}\\
      \text{sampled impulse response: } h(kT) = -2e^{-6kT} + 2e^{-kT},\;k\in\N\\
      \text{$z$-transfer function: } H(z) = \frac{-2z}{z-e^{-6T}} + \frac{2z}{z-e^{-T}} = \frac{2z(e^{-T} - e^{-6T})}{(z-e^{-6T})(z-e^{-T})}
    \end{align*} 
    \item If the systems are separated by a sampler
    \begin{align*}
      H_1(z) = \frac{z}{z-e^{-6T}}\\
      H_2(z) = \frac{10z}{z-e^{-T}}\\
      \text{$z$-transfer function: } H(z) = \frac{10z^2}{(z-e^-6T)(z-e^{-T})}\\
      \text{partial fraction expansion: }\frac{10z^2}{(z-e^{-6T})(z-e^{-T})} = \frac{Az}{z-e^{-6T}}+\frac{Bz}{z-e^{-T}}\\
      10z = (z-e^{-T})A + (z-e^{-6T})B\\
      z=e^{-6T} \implies 10e^{-6T} = (e^{-6T} - e^{-T})A \implies A = \frac{10e^{-6T}}{e^{-6T}-e^{-T}}\\
      z=e^{-T} \implies 10e^{-T} = (e^{-T} - e^{-6T})B \implies B = \frac{10e^{-T}}{e^{-T} - e^{-6T}}\\
      H(z) = \frac{10e^{-6T}z}{(e^{-6T}-e^{-T})(z-e^{-6T})}+\frac{10e^{-T}z}{(e^{-T}-e^{-6T})(z-e^{-T})}\\
      = \frac{10}{e^{-6T}-e^{-T}}\left(\frac{e^{-6T}z}{z-e^{-6T}}-\frac{e^{-T}z}{z-e^{-T}}\right)\\
      \text{sampled impulse response: } h(kT) = \frac{10}{e^{-6T} - e^{-T}}(e^{-6(k+1)T}-e^{-(k+1)T}),\;k\in\N
    \end{align*} 
  \end{enumerate}
\end{problem}
% --------------------------------------------------------------
%     You don't have to mess with anything below this line.
% --------------------------------------------------------------
 
\end{document}