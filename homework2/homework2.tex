%%%%%%%%%%%%%%%%%%%%%%%%%%%%%%%%%%%%%%%%%%%%%%%%%%%%%%%%%%%%%%%
%
% Welcome to writeLaTeX --- just edit your LaTeX on the left,
% and we'll compile it for you on the right. If you give
% someone the link to this page, they can edit at the same
% time. See the help menu above for more info. Enjoy!
%
%%%%%%%%%%%%%%%%%%%%%%%%%%%%%%%%%%%%%%%%%%%%%%%%%%%%%%%%%%%%%%%

% --------------------------------------------------------------
% This is all preamble stuff that you don't have to worry about.
% Head down to where it says "Start here"
% --------------------------------------------------------------
 
\documentclass[12pt]{article}
 
\usepackage[margin=1in]{geometry}
\usepackage{amsmath,amsthm,amssymb}

\usepackage{listings}
\usepackage{xcolor}
\usepackage{circuitikz}
\usetikzlibrary{bending}
\usetikzlibrary{patterns,decorations.pathmorphing,positioning}
\usepackage{enumitem}

%New colors defined below
\definecolor{codegreen}{rgb}{0,0.6,0}
\definecolor{codegray}{rgb}{0.5,0.5,0.5}
\definecolor{codepurple}{rgb}{0.58,0,0.82}
\definecolor{backcolour}{rgb}{0.95,0.95,0.92}

%Code listing style named "mystyle"
\lstdefinestyle{mystyle}{
  backgroundcolor=\color{backcolour}, commentstyle=\color{codegreen},
  keywordstyle=\color{magenta},
  numberstyle=\tiny\color{codegray},
  stringstyle=\color{codepurple},
  basicstyle=\ttfamily\footnotesize,
  breakatwhitespace=false,         
  breaklines=true,                 
  captionpos=b,                    
  keepspaces=true,                 
  numbers=left,                    
  numbersep=5pt,                  
  showspaces=false,                
  showstringspaces=false,
  showtabs=false,                  
  tabsize=2
}

%"mystyle" code listing set
\lstset{style=mystyle}

 
\newcommand{\N}{\mathbb{N}}
\newcommand{\Z}{\mathbb{Z}}
 
\newenvironment{theorem}[2][Theorem]{\begin{trivlist}
\item[\hskip \labelsep {\bfseries #1}\hskip \labelsep {\bfseries #2.}]}{\end{trivlist}}
\newenvironment{lemma}[2][Lemma]{\begin{trivlist}
\item[\hskip \labelsep {\bfseries #1}\hskip \labelsep {\bfseries #2.}]}{\end{trivlist}}
\newenvironment{exercise}[2][Exercise]{\begin{trivlist}
\item[\hskip \labelsep {\bfseries #1}\hskip \labelsep {\bfseries #2.}]}{\end{trivlist}}
\newenvironment{problem}[2][Problem]{\begin{trivlist}
\item[\hskip \labelsep {\bfseries #1}\hskip \labelsep {\bfseries #2.}]}{\end{trivlist}}
\newenvironment{question}[2][Question]{\begin{trivlist}
\item[\hskip \labelsep {\bfseries #1}\hskip \labelsep {\bfseries #2.}]}{\end{trivlist}}
\newenvironment{corollary}[2][Corollary]{\begin{trivlist}
\item[\hskip \labelsep {\bfseries #1}\hskip \labelsep {\bfseries #2.}]}{\end{trivlist}}

\newenvironment{solution}{\begin{proof}[Solution]}{\end{proof}}
 
\begin{document}
 
% --------------------------------------------------------------
%                         Start here
% --------------------------------------------------------------
 
\title{Homework 2}%replace X with the appropriate number
\author{Mengxiang Jiang\\ %replace with your name
EEEN 5338 Digital and DSP Based Control} %if necessary, replace with your course title
 
\maketitle
 
\begin{problem}{1} %You can use theorem, exercise, problem, or question here.  Modify x.yz to be whatever number you are proving
    For each of the following equations, determine the order of the
    equation and then test it for (i) linearity, (ii) time invariance, and (iii)
    homogeneity.
    \begin{enumerate}[label=\alph*.]
        \item $y(k+2) = y(k+1)y(k) + u(k)$\\
        This equation is 2nd order from the difference of $k+2$ and $k$.\\
        It is not linear since $y(k+1)y(k)$ is not linear.\\
        It is time invariant because all the coefficients are constant.\\
        Since $u(k)$ is not 0, it is not homogeneous.

        \item $y(k+3) + 2y(k) = 0$\\
        This equation is 3rd order from the difference of $k+3$ and $k$.\\
        It is linear, since all the functions are linear.\\
        It is time invariant because all the coefficients are constant.\\
        Since $u(k)$ is 0, it is homogeneous.
        
        \item $y(k+4) + y(k-1) = u(k)$\\
        This equation is 5th order from the difference of $k+4$ and $k-1$.\\
        It is linear, since all the functions are linear.\\
        It is time invariant because all the coefficients are constant.\\
        Since $u(k)$ is not 0, it is not homogeneous.

        \item $y(k+5) = y(k+4) + u(k+1) - u(k)$\\
        This equation is 5th order from the difference of $k+5$ and $k$.\\
        It is linear, since all the functions are linear.\\
        It is time invariant because all the coefficients are constant.\\
        Since $u(k)$ is not 0, it is not homogeneous.

        \item $y(k+2) = y(k)u(k)$\\
        This equation is 2nd order from the difference of $k+2$ and $k$.\\
        It is not linear, since $y(k)u(k)$ is not linear.\\
        It is time invariant because all the coefficients are constant.\\
        Since $u(k)$ is not 0, it is not homogeneous.
    \end{enumerate}

\end{problem}
\pagebreak
\begin{problem}{2} %You can use theorem, exercise, problem, or question here.  Modify x.yz to be whatever number you are proving
    Solve the following difference equations. 
    \begin{enumerate}[label=\alph*.]
        \item $y(k+1)-0.8y(k)=0,\; y(0)=1$\\
        \begin{align*}
            zY(z)-z-0.8Y(z)=0\\
            (z-0.8)Y(z)=z\\
            Y(z) = \frac{z}{z-0.8}\\
            f(k)=(0.8)^k,\;k\in \N
        \end{align*}
        \item $y(k+1)-0.8y(k)=1(k),\; y(0)=0$\\
        \begin{align*}
            zY(z)-0.8Y(z)=\frac{z}{z-1}\\
            Y(z)=\frac{z}{(z-0.8)(z-1)}\\
            z\left(\frac{1}{(z-0.8)(z-1)}\right)=z\left(\frac{A}{z-0.8}+\frac{B}{z-1}\right)\\
            1 = (z-1)A + (z-0.8)B\\
            z=1\implies 1=0.2B \implies B=5\\
            z=0.8\implies 1=-0.2A \implies A=-5\\
            Y(z)=\frac{-5z}{z-0.8} + \frac{5z}{z-1}\\
            f(k)=-5(0.8)^k + 5,\;k\in \N
        \end{align*}
        \item $y(k+1)-0.8y(k)=1(k),\; y(0)=1$\\
        \begin{align*}
            zY(z)-z-0.8Y(z)=\frac{z}{z-1}\\
            Y(z)=\frac{z}{z-0.8}+\frac{z}{(z-0.8)(z-1)}\\
            \text{from partial fraction decomposition of part b we get:}\\
            Y(z)=\frac{z}{z-0.8}+\frac{-5z}{z-0.8} + \frac{5z}{z-1}=\frac{-4z}{z-0.8} + \frac{5z}{z-1}\\
            f(k)=-4(0.8)^k + 5,\;k\in \N
        \end{align*}
        \pagebreak
        \item $y(k+2)+0.7y(k+1)+0.06y(k)=\delta(k),\; y(0)=0,\;y(1)=2$\\
        \begin{align*}
            z^2Y(z)+0.7zY(z)+0.06Y(z)=1+2z\\
            Y(z)=\frac{2z+1}{z^2+0.7z+0.06}=\frac{2z+1}{(z+0.1)(z+0.6)}\\
            \frac{2z+1}{(z+0.1)(z+0.6)} = \frac{A}{z+0.1}+\frac{B}{z+0.6}\\
            2z+1=(z+0.6)A+(z+0.1)B\\
            z=-0.6\implies -0.2 = -0.5B \implies B=0.4\\
            z=-0.1\implies 0.8=0.5A \implies A = 1.6\\
            Y(z) = \frac{1.6}{z+0.1}+\frac{0.4}{z+0.6}\\
            \text{this is not in the table, but we can try decomposing $\frac{Y(z)}{z}$ instead:}\\
            \frac{Y(z)}{z}=\frac{2z+1}{z(z+0.1)(z+0.6)}\\
            \frac{2z+1}{z(z+0.1)(z+0.6)} = \frac{A}{z} + \frac{B}{z+0.1}+\frac{C}{z+0.6}\\
            2z+1=(z+0.1)(z+0.6)A+z(z+0.6)B+z(z+0.1)C\\
            z=0\implies 1 = 0.06A \implies A = \frac{100}{6}=\frac{50}{3}\\
            z=-0.1\implies 0.8 = -0.05B \implies B = -16\\
            z=-0.6\implies -0.2 = 0.3C\implies C = -\frac{2}{3}\\
            \frac{Y(z)}{z}=\frac{\frac{50}{3}}{z} + \frac{-16}{z+0.1} + \frac{-\frac{2}{3}}{z+0.6}\\
            Y(z)=\frac{50}{3} +\frac{-16z}{z+0.1}+\frac{-\frac{2}{3}z}{z+0.6}\\
            f(k)=\frac{50}{3}\delta(k) -16(-0.1)^k-\frac{2}{3}(-0.6)^k,\;k\in \N
        \end{align*}
    \end{enumerate}
\end{problem}
\pagebreak
\begin{problem}{3}
    Given the discrete-time system
    $$y(k+2)-y(k)=2u(k)$$
    find the impulse response of the system $g(k)$:
    \begin{enumerate}[label=\alph*.]
        \item From the difference equation
        \begin{align*}
            k = -2 \implies y(0) - y(-2) = 0,\;\text{assuming y is causal}\implies y(0) = 0\\
            k = -1 \implies y(1) - y(-1) = 0 \implies y(1) = 0\\
            k = 0 \implies y(2) - y(0) = 2 \implies y(2) = 2\\
            k = 1 \implies y(3) - y(1) = 0 \implies y(3) = y(1) = 0\\
            k = 2 \implies y(4) - y(2) = 0 \implies y(4) = y(2) = 2\\
            g(k) = \begin{cases}
                2 & \text{if $k > 0$ and $k$ is even}\\
                0 & \text{otherwise}
            \end{cases}
        \end{align*}
        \item Using z-transformation
        \begin{align*}
            z^2Y(z) - Y(z) = 2U(z)\\
            G(z) = \frac{Y(z)}{U(z)} = \frac{2}{z^2-1} = \frac{2}{(z+1)(z-1)}\\
            \frac{G(z)}{z} = \frac{2}{z(z+1)(z-1)} = \frac{A}{z}+\frac{B}{z+1}+\frac{C}{z-1}\\
            2=(z+1)(z-1)A + z(z-1)B + z(z+1)C\\
            z=0 \implies 2 = -A \implies A=-2\\
            z=-1 \implies 2 =2B \implies B=1\\
            z=1 \implies 2 = 2C \implies C=1\\
            \frac{G(z)}{z} = \frac{-2}{z} + \frac{1}{z+1}+\frac{1}{z-1}\\
            G(z) = -2 + \frac{z}{z+1} + \frac{z}{z-1}\\
            g(k) = \begin{cases}
                -2\delta(k) + (-1)^k + 1 & k\in \N\\
                0 & k < 0
            \end{cases} 
        \end{align*}
    \end{enumerate}
\end{problem}
\pagebreak
\begin{problem}{4}
    Find the impulse response functions for the systems governed by the
    following difference equations.
    \begin{enumerate}[label=\alph*.]
        \item $y(k+1) - 0.5y(k)=u(k)$
        \begin{align*}
            zY(z)-0.5Y(z)=U(z)\\
            G(z) = \frac{Y(z)}{U(z)} = \frac{1}{z-0.5}=z^{-1}\frac{z}{z-0.5}\\
            g(k) = \begin{cases}
                (0.5)^{k-1} & k \geq 1\\
                0 & k<1
            \end{cases}
        \end{align*}
        \item $y(k+2) - 0.1 y(k+1) + 0.8 y(k) = u(k)$
        \begin{align*}
            z^2Y(z)-0.1zY(z) + 0.8Y(z)=U(z)\\
            G(z) = \frac{Y(z)}{U(z)} = \frac{1}{z^2-0.1z+0.8}\\
            \text{does not look easily factorable, using the quadratic formula to find roots: }\\
            z = \frac{0.1\pm\sqrt{0.01-3.2}}{2} = \frac{0.1\pm j\sqrt{3.19}}{2}\\
            \text{the roots are complex conjugates, using $a^k\sin{bk}$ $z$-transform:}\\
            \mathcal{Z}\{a^k\sin{bk}\} = \frac{az\sin{b}}{z^2-2az\cos{b}+a^2}\\
            G(z) = z^{-1}\frac{z}{z^2-0.1z+0.8}\\
            \text{and equating coefficients:}\\
            a=\sqrt{0.8} \approx 0.894\\
            \cos{b} = \frac{0.05}{\sqrt{0.8}} \approx 0.056\\
            b = \cos^{-1}\left({\frac{0.05}{\sqrt{0.8}}}\right) \approx 1.515\\
            \sin{b} \approx 0.998\\
            \frac{1}{a\sin{b}} \approx 1.12\\
            \implies g(k) = \begin{cases}
                1.12(0.894)^{k-1}\sin(1.515(k-1)) & k \geq 1\\
                0 & k < 1
            \end{cases}
        \end{align*}
    \end{enumerate}
\end{problem}
\pagebreak
\begin{problem}{5}
    Find the steady-state response of the systems resulting from the
    sinusoidal input $u(k) = 0.5 \sin(0.4 k)$.
    \begin{enumerate}[label=\alph*.]
        \item $$H(z) = \frac{z}{z-0.4}$$
        \begin{align*}
            H(e^{0.4j}) = \frac{e^{0.4j}}{e^{0.4j}-0.4} \approx 1.537e^{-0.242j}\\
            y_{ss}(k)\approx 0.5(1.537)\sin(0.4k-0.242) \approx 0.769\sin(0.4k-0.242)
        \end{align*} 
        \item $$H(z) = \frac{z}{z^2+0.4z+0.03}$$
        \begin{align*}
            H(e^{0.4j}) = \frac{e^{0.4j}}{e^{0.8j}+0.4e^{0.4j}+0.03} \approx 0.714e^{-0.273j}\\
            y_{ss}(k)\approx 0.5(0.714)\sin(0.4k-0.273) \approx 0.357\sin(0.4k-0.273)
        \end{align*}
    \end{enumerate}
\end{problem}
% --------------------------------------------------------------
%     You don't have to mess with anything below this line.
% --------------------------------------------------------------
 
\end{document}